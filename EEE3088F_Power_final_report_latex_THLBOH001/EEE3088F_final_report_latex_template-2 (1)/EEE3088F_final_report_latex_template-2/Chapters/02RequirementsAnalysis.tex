% ----------------------------------------------------
% Requirements Analysis
% ----------------------------------------------------
\documentclass[class=report,11pt,crop=false]{standalone}
\input{../Style/ChapterStyle.tex}
\begin{document}
% ----------------------------------------------------
\chapter{Requirements Analysis} \label{ch:reqAnalysis}
\vspace{-1cm}
% ----------------------------------------------------

\section{Requirements}
The requirements for a micromouse power module are described in \autoref{tab:requirements}.
\begin{table}[h]
    \centering
    \caption{User and functional requirements of the power subsystem.}    \label{tab:requirements}
    \begin{tabular}{|c|m{12cm}|}
        \hline
        \textbf{ Requirement ID} & \textbf{Description} \\
        \hline
         UR01 & Drive the two motors\\
         \hline
         UR02 & Has an analog connection to indicate to the processor the battery's voltage. This is to detect the battery state of charge \\
         \hline
         UR03 & Charge the battery\\
         \hline
         UR04 & ON/OFF switch \\
         \hline
         UR05 & Include a connector for the battery \\
         \hline
         UR06 & Include connector for the pin header \\
         \hline
         UR07 & Board must be of an appropriate size and shape \\
         \hline
         UR04 & Adhere to the budget \\
         \hline
    \end{tabular}
\end{table}

% =====================================================
\section{Specifications}
The specifications, refined from the requirements in \autoref{tab:requirements}, for the micromouse power module are described in \autoref{tab:specifications}.
\begin{table}[h]
    \centering
    \caption{Specifications of the power subsystem derived from the requirements in \autoref{tab:requirements}.}    \label{tab:specifications}
    \begin{tabular}{|c|m{12cm}|}
        \hline
        \textbf{Specification ID} & \textbf{Description} \\
        \hline
         SP01 & Each motor has to only draw a current of 200mA at the maximum voltage of the batter (around 4.2V) \\
         \hline
         SP02 & Charges battery from 5V input pin \\
         \hline
         SP03 & Use a 1S1P battery \\
         \hline
         SP04 & Use the 5V input pin to charge the battery\\
         \hline
         SP05 & At OFF state: the battery should draw less that 500uA \\
         \hline
         SP06 & At ON state: should be at the micromouse's peak current \\
         \hline
         SP07 & Use a JST PH 2mm pin pitch connector for the battery \\
         \hline
         SP08 & Use a 2x8 (2.54mm pin pitch) pin header \\
         \hline
         SP09 & shape of the PCB: height- 18mm or greater, width- 35mm or less. Connector pins should be at the middle \\
         \hline
         SP010 & Construction should be within the budget of R360 \\
         \hline
    \end{tabular}
\end{table}

% =====================================================
\section{Testing Procedures}
A summary of the testing procedures detailed in \autoref{ch:atp} is given in \autoref{tab:atps_summary}.
\begin{table}[h]
    \centering
    \caption{ The summary of acceptance tests that will be used derived form the acceptance tests chapter}
    \label{tab:atps_summary}
    \begin{tabular}{|c|m{10cm}|}
        \hline
        \textbf{Acceptance Test ID} & \textbf{Description} \\
        \hline
         AT01 & Current drawn from each of the motors [using an ammeter]\\
         \hline
         AT02 & Battery's  output voltage to sense the state of charge [voltmeter]\\
         \hline
         AT03 & Current drawn at ON stage [Ammeter]\\
         \hline
         AT04 & Current drawn at OFF stage [Ammeter]\\
         \hline
         AT05 & Ensure the battery is being charged [Voltmeter]\\
         \hline
    \end{tabular}
\end{table}

% =====================================================
\section{Traceability Analysis}
The show how the requirements, specifications and testing procedures all link, \autoref{tab:matrix} is provided.

\begin{table}[h]
    \centering
    \caption{Requirements Traceability Matrix}
    \label{tab:matrix}
    \begin{tabular}{|c|c|c|c|}
        \hline
        \# & Requirements & Specifications  & Acceptance Test\\
        \hline
         1 & UR01 &  SP01 & AT01 \\
         2 & FR01 & SP02,SP05 & AT03 \\
         & &  & \\
    \hline
    \end{tabular}
\end{table}

\subsection{Traceability Analysis 1}
UR01 is this from which SP01, blah blah blah, can be derived. To test this AT01 is suggested because blah blah blah.

\subsection{Traceability Analysis 2}
From FR02, yadda yadda, SP02 and SP05 can be derived because blah blah blah. These can be tested through AT03 which tests yadda yadda yadda.


% ----------------------------------------------------
\ifstandalone
\bibliography{../Bibliography/References.bib}
\fi
\end{document}
% ----------------------------------------------------